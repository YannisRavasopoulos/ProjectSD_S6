\subsection{Edit Profile}
\label{uc:edit-profile}

Ο χρήστης επιθυμεί να ενημερώσει τα προσωπικά του στοιχεία.

\subsubsection{Βασική Ροή}

\begin{enumerate}
    \item[1] Ο χρήστης μπαίνει στην ενότητα "Profile".
    \item[2] Το σύστημα εμφανίζει τα τρέχοντα προσωπικά στοιχεία.
    \item[3] Ο χρήστης τροποποιεί ένα ή περισσότερα στοιχεία.
    \item[4] Ο χρήστης επιλέγει "Confirm".
    \item[5] Το σύστημα εκτελεί έλεγχο τον στοιχείων.
    \item[6] To σύστημα ενημερώνει τα στοιχεία και αποθηκεύει τις αλλαγές.
\end{enumerate}

\subsubsection{Εναλλακτική Ροή: Λανθασμένα στοιχεία}

\begin{enumerate}
    \item[6] Το σύστημα εμφανίζει σφάλμα και ζητά διόρθωση.
    \item[7] Ο χρήστης διορθώνει τα στοιχεία.
    \item[8] Συνέχεια από το βήμα 4 της βασικής ροής.
\end{enumerate}

\subsubsection{Εναλλακτική Ροή: Ακύρωση}

\begin{enumerate}
    \item[4] Ο χρήστης εγκαταλείπει την ενότητα Profile.
    \item[5] Το σύστημα δεν ενημερώνει τα στοιχεία και απορρίπτει τυχόν αλλαγές.
\end{enumerate}




\subsection{Report User}

Ο χρήστης επιθυμεί να αναφέρει έναν άλλο χρήστη της εφαρμογής για
παράνομη ή ανεπιθύμητη δραστηριότητα.

\subsubsection{Βασική Ροή}

\begin{enumerate}
    \item Ο χρήστης επιλέγει τον χρήστη που θέλει να αναφέρει και πατάει "Report".
    \item H εφαρμογή εμφανίζει την φόρμα αναφοράς.
    \item Ο χρήστης επιλέγει τον λόγο αναφοράς, προσθέτει σχόλια και πατάει υποβολή.
    \item Η εφαρμογή ενημερώνει τον χρήστη για την επιτυχή υποβολή.
    \item Το σύστημα επεξεργάζεται την αναφορά και την τοποθετεί σε λίστα αναμονής.
\end{enumerate}

\subsubsection{Εναλλακτική Ροή: Ακύρωση}

\begin{enumerate}
    \item[3] Ο χρήστης αποφασίζει να μην υποβάλει την αναφορά και πατάει "Cancel".
\end{enumerate}


\subsection{Redeem Reward}

Ο χρήστης επιθυμεί να λάβει μια ανταμοιβή που έχει κερδίσει μέσω
της εφαρμογής.

\subsubsection{Βασική Ροή}

\begin{enumerate}
    \item Ο χρήστης επιλέγει "Redeem Reward"
    \item Η εφαρμογή υπολογίζει τους πόντους του χρήστη.
    \item Η εφαρμογή εμφανίζει τις διαθέσιμες ανταμοιβές και τους πόντους.
    \item Ο χρήστης επιλέγει την ανταμοιβή που επιθυμεί.
    \item Η εφαρμογή ενημερώνει τους πόντους του χρήστη και αφαιρεί την ανταμοιβή
          από τη λίστα των διαθέσιμων ανταμοιβών.
    \item Η εφαρμογή εμφανίζει τον κωδικό εξαργύρωσης.
\end{enumerate}

\subsubsection{Εναλλακτική Ροή: Ακύρωση}

\begin{enumerate}
    \item[4] Ο χρήστης εγκαταλείπει την διαδικασία.
\end{enumerate}

\subsubsection{Εναλλακτική Ροή: Δεν υπάρχουν διαθέσιμες ανταμοιβές}

\begin{enumerate}
    \item[3] Η εφαρμογή εμφανίζει μήνυμα ότι δεν υπάρχουν διαθέσιμες ανταμοιβές.
\end{enumerate}

\subsubsection{Εναλλακτική Ροή: Ανεπαρκής αριθμός πόντων}

\begin{enumerate}
    \item[5] Η εφαρμογή εμφανίζει μήνυμα ότι ο χρήστης δεν έχει αρκετούς πόντους.
    \item[6] Συνέχεια από το βήμα 3 της βασικής ροής.
\end{enumerate}

\subsection{Report Parking}

Ο χρήστης επιθυμεί να αναφέρει παράνομη ή ανεπιθύμητη στάθμευση.

\subsubsection{Βασική Ροή}

\begin{enumerate}
    \item Ο χρήστης ανοίγει την εφαρμογή και επιλέγει "Report Parking".
    \item Η εφαρμογή εμφανίζει την φόρμα αναφοράς.
    \item Ο χρήστης συμπληρώνει την φόρμα με τα στοιχεία της αναφοράς
          και επιλέγει "Submit".
    \item Το σύστημα αποθηκεύει την αναφορά και ενημερώνει τον χρήστη.
    \item Το σύστημα εκτελεί έλεγχο για την εγκυρότητα των στοιχείων.
    \item Διαπιστώνεται η εγκυρότητα της αναφοράς και ενημερώνεται η τροχαία
\end{enumerate}

\subsubsection{Εναλλακτική Ροή}

\begin{enumerate}
    \item[6] Διαπιστώνεται η μη εγκυρότητα της αναφοράς και μεταφέρεται στο αρχείο.
\end{enumerate}

\subsubsection{Εναλλακτική Ροή}

\begin{enumerate}
    \item[2] Ο χρήστης ακυρώνει την αναφορά.
    \item[3] Η εφαρμογή επιστρέφει στην αρχική οθόνη.
\end{enumerate}
