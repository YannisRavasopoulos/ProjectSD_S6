\subsubsection{Edit Activity}

\begin{enumerate}
    \item Ο χρήστης επιλέγει "Edit Activity".
    \item Η εφαρμογή εμφανίζει μενού με επιλογές για την ιδιότητα του χρήστη (πχ φοιτητής).
    \item Ο χρήστης επιλέγει την ιδιότητα του.
    \item Η εφαρμογή εμφανίζει την φόρμα αναζήτησης και έναν χάρτη της περιοχής του χρήστη.
    \item Ο χρήστης δηλώνει την περιοχή στην οποία επιθυμεί να μετακινηθεί.
    \item Η εφαρμογή εμφανίζει μενού με επιλογές για τις μέρες και τις ώρες έναρξης και λήξης
          της δραστηριότητας.
    \item Ο χρήστης εισάγει τα κατάλληλα στοιχεία.
    \item H εφαρμογή εμφανίζει μενού με επιλόγες για το μέσο μεταφοράς του χρήστη
    \item Ο χρήστης επιλέγει το μέσο μεταφοράς του.
    \item Το σύστημα εκτελεί προεπεξεργασία στα δεδομένα.
    \item Το σύστημα εισάγει την δραστηριότητα στο κατάλογο δραστηριοτήτων του χρήστη.
    \item Η εφαρμογή εμφανίζει μήνυμα επιτυχίας.
\end{enumerate}

\subsubsection{Εναλλακτική Ροή: Ακύρωση 1}

\begin{enumerate}
    \item[3] Ο χρήστης ακυρώνει την διαδικασία και επιστρέφει στην αρχική οθόνη.
\end{enumerate}

\subsubsection{Εναλλακτική Ροή: Ακύρωση 2}

\begin{enumerate}
    \item[5] Ο χρήστης ακυρώνει την διαδικασία και επιστρέφει στην αρχική οθόνη.
\end{enumerate}

\subsubsection{Εναλλακτική Ροή: Ακύρωση 3}

\begin{enumerate}
    \item[7] Ο χρήστης ακυρώνει την διαδικασία και επιστρέφει στην αρχική οθόνη.
\end{enumerate}

\subsubsection{Εναλλακτική Ροή: Ακύρωση 4}

\begin{enumerate}
    \item[9] Ο χρήστης ακυρώνει την διαδικασία και επιστρέφει στην αρχική οθόνη.
\end{enumerate}

\subsubsection{Εναλλακτική Ροή: Μη έγκυρα στοιχεία}

\begin{enumerate}
    \item[7] Ο χρήστης εισάγει μη έγκυρα στοιχεία.
    \item[8] Το σύστημα εμφανίζει μήνυμα σφάλματος και ζητά διόρθωση.
    \item[9] Συνέχεια από το βήμα 6 της βασικής ροής.
\end{enumerate}
