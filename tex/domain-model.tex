\documentclass[11pt]{article}

\usepackage[a4paper,margin=3cm]{geometry} % For page dimensions
\usepackage{fontspec} % For font selection
\usepackage{unicode-math} % For mathematical fonts
\usepackage{polyglossia} % For language selection
\usepackage{graphicx} % For images
\usepackage[colorlinks=true, linkcolor=black, urlcolor=blue, citecolor=green]{hyperref} % For hyperlinks
\usepackage{xcolor} % Colors for code highlighting
\usepackage{fancyhdr} % For headers and footers
\usepackage{bookmark}
\usepackage{nameref} % For referencing sections
\usepackage{draftwatermark} % For watermarks
\usepackage{amsmath} % For mathematical equations
\usepackage{listings} % For code formattings
\usepackage{tikz} % For diagrams

\graphicspath{ {images} }

% Set fonts
% \setmainfont{Noto Serif}
\setromanfont{Noto Serif}
\setsansfont{Noto Sans}
\setmonofont{Noto Sans Mono}

% Set languages
\setmainlanguage{greek}
\setotherlanguages{english}

% Header and footer settings
\pagestyle{fancy}
\setlength{\headheight}{14pt}
\fancyhf{}
\fancyfoot[C]{\thepage}

% Custom Commands
\newcommand{\email}[1]{\href{mailto://#1}{\texttt{#1}}} % Email formatting
\newcommand{\developer}[2]{#1 (#2) \\ \email{up#2@ac.upatras.gr} \\[2ex]}
% TODO
\newcommand{\appname}{APPNAME}


\author{
    \developer{Γιάννης Ραβασόπουλος}{1100696}
    \developer{Κώστας Λουκανάρης}{1100610}
    \developer{Χρήστος Μάριος Νικολόπουλος}{1100644}
    \developer{Άγγελος Αβεντισιάν}{1100491}
    \developer{Βασίλης Μυλωνάς}{1100643}
}

\date{
    \today \\[1ex]
    Έκδοση 0.1 \\
}


\fancyhead[L]{Domain Model}
\fancyhead[R]{\leftmark}

\title{
    Domain Model - \appname\\[1ex]
    \large Τεχνολογία Λογισμικού - ΤΜΗΥΠ, Πανεπιστήμιο Πατρών \\[2ex]
}

\begin{document}

\maketitle
\thispagestyle{empty}
\newpage

\tableofcontents
\newpage

\begin{abstract}
    Περιγραφή των βασικών οντοτήτων και σχέσεων της εφαρμογής \appname,
\end{abstract}

\newpage

\section{Ορολογία}

\begin{description}
    \item[Χρήστης (User)] \hfill \\
        Οποιοσδήποτε χρησιμοποιεί την εφαρμογή.

    \item[Οδηγός (Driver)] \hfill \\
        Χρήστης ο οποίος διαθέτει όχημα και προσφέρει δυνατότητες μετακίνησης σε άλλους χρήστες.

    \item[Συνεπιβάτης (Carpooler)] \hfill \\
        Χρήστης ο οποίος επιθυμεί να λάβει μεταφορά από κάποιον οδηγό.

    \item[Όχημα (Vehicle)] \hfill \\
        Το μέσο μεταφοράς που χρησιμοποιείται από τον οδηγό, περιλαμβάνοντας πληροφορίες όπως
        μοντέλο, αριθμός επιβατών και ο αριθμός κυκλοφορίας.

    \item[Βαθμολογία (Rating)] \hfill \\
        Η αξιολόγηση που δίνεται από ένας χρήστης σε έναν άλλο. Μπορεί να περιλαμβάνει
        σχόλια.

    \item[Αναφορά (Report)] \hfill \\
        Τρόπος κατά τον οποίο ένας χρήστης καταγγέλλει παράνομη ή ανεπιθύμητη συμπεριφορά.
        Περιλαμβάνει τον λόγο της καταγγελίας και περιγραφή.

    \item[Τοποθεσία (Location)] \hfill \\
        Μια γεωγραφική τοποθεσία όπως σημείο αναχώρησης ή άφιξης.

    \item[Διαδρομή (Route)] \hfill \\
        Η πορεία από μια τοποθεσία σε μια άλλη.

    \item[Ανταμοιβή (Reward)] \hfill \\
        Κάποιου είδους έπαθλο για τους χρήστες της εφαρμογής. Συνήθως με τη μορφή κουπονιού,
        το οποίο μπορεί να εξαργυρωθεί εντός της εφαρμογής.

    \item[Συνεπιβίβαση (Ride)] \hfill \\
        Μια διαδικασία όπου ο οδηγός μεταφέρει έναν ή περισσότερους συνεπιβιβάτες.

    \item[Παραλαβή (Pickup)] \hfill \\
        Η ενέργεια κατά την οποία ένας συνεπιβάτης εισέρχεται στο όχημα του οδηγού.

    \item[Δραστηριότητα (Activity)] \hfill \\
        Μια ενέργεια που επιθυμεί να πραγματοποιήσει ο χρήστης, σε συγκεκριμένο τόπο και χρόνο, για
        την οποία απαιτείται μεταφορικό μέσο.

    \item[InstaRide] \hfill \\
        Μια υπηρεσία που προσφέρει η εφαρμογή, η οποία επιτρέπει στους χρήστες να
        πραγματοποιούν διαδρομές χωρίς προγραμματισμό. Ειδική περίπτωση του Ride.
\end{description}

\begin{figure}
    \centering
    \includegraphics[width=\textwidth]{uml/domain-model}
    \caption{Domain Model}
\end{figure}

\end{document}
