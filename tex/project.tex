\documentclass[11pt]{article}

\usepackage[a4paper,margin=3cm]{geometry} % For page dimensions
\usepackage{fontspec} % For font selection
\usepackage{unicode-math} % For mathematical fonts
\usepackage{polyglossia} % For language selection
\usepackage{graphicx} % For images
\usepackage[colorlinks=true, linkcolor=black, urlcolor=blue, citecolor=green]{hyperref} % For hyperlinks
\usepackage{xcolor} % Colors for code highlighting
\usepackage{fancyhdr} % For headers and footers
\usepackage{bookmark}
\usepackage{nameref}
% \usepackage{draftwatermark} % For watermarks

\usepackage{amsmath} % For mathematical equations
\usepackage{listings} % For code formattings
\usepackage{tikz} % For diagrams

\graphicspath{ {images} }

% Set fonts
% \setmainfont{Noto Serif}
\setromanfont{Noto Serif}
\setsansfont{Noto Sans}
\setmonofont{Noto Sans Mono}

% Set languages
\setmainlanguage{greek}
\setotherlanguages{english}

% Watermark
% \SetWatermarkScale{1}
% \SetWatermarkColor{gray}
% \SetWatermarkLightness{0.5}

% Header and footer settings
\pagestyle{fancy}
\setlength{\headheight}{14pt}
\fancyhf{}
\fancyhead[L]{ParkOut}
\fancyhead[R]{\leftmark} % section name
\fancyfoot[C]{\thepage}

\newcommand{\email}[1]{\href{mailto://#1}{\texttt{#1}}} % Email formatting

\begin{document}
\begin{titlepage}
    \centering
    \Huge
    ParkOut \\
    \normalsize
    \vspace{1cm}
    % TODO project logo
    % \includegraphics[width=0.5\textwidth]{upatras-logo.jpg}
    % \vspace{1cm}
    \begin{tabular}{cc}
        Γιάννης Ραβασόπουλος            & Κώστας Λουκανάρης               \\
        1100696                         & 1100610                         \\
        \email{up1100696@ac.upatras.gr} & \email{up1100610@ac.upatras.gr} \\
        \\
        Χρήστος Μάριος Νικολόπουλος     & Άγγελος Αβεντισιάν              \\
        1100644                         & 1100491                         \\
        \email{up1100644@ac.upatras.gr} & \email{up1100491@ac.upatras.gr} \\
        \\
    \end{tabular} \\
    Βασίλης Μυλωνάς \\
    1100643 \\
    \email{up1100643@ac.upatras.gr} \\
    \vspace{1.5cm}
    \Large
    \today \\
    \vspace{0.5cm}
    Τεχνολογία Λογισμικού \\
    \vspace{0.5cm}
    Τμήμα Μηχανικών Η/Υ και Πληροφορικής, \\
    Πανεπιστήμιο Πατρών \\
    \normalsize
\end{titlepage}

\newpage

\begin{abstract}
    Ανάπτυξη συστήματος και συνοδεύουσας εφαρμογής για την διευκόλυνση
    συνεπιβίβασης σε κοινές διαδρομές (carpooling).
    Η εφαρμογή εγκαθισταται σε κινητές συσκευές και παρέχει ζωντανή πληροφόρηση
    αντλώντας πληροφορίες από τους χρήστες. H εφαρμογή απευθύνεται αμφότερα σε
    οδηγούς αυτοκινήτων αλλά και σε πεζούς. Στόχος είναι η καλύτερη αξιοποίηση
    του οδικού δικτύου και η μείωση της κυκλοφοριακής συμφόρησης σε αστικές
    περιοχές.
\end{abstract}

\newpage

\tableofcontents

\newpage

\section{Σύσταση Ομάδας}

Η ομάδα αποτελείται από 5 μέλη:

\begin{itemize}
    \item Βασίλης Μυλωνάς (Επικεφαλής Έργου/Project Manager)
    \item Άγγελος Αβεντισιάν (Υπεύθυνος Ποιότητας/QA Manager)
    \item Γιάννης Ραβασόπουλος
    \item Κώστας Λουκανάρης
    \item Χρήστος Μάριος Νικολόπουλος
\end{itemize}

Η επιλογή των ρόλων έγινε με βάση την εμπειρία και τα ενδιαφέροντα του κάθε
μέλους έπειτα από ψηφοφορία.

\newpage

\section{Περιγραφή}

\subsection{Το Πρόβλημα της Κυκλοφοριακής Συμφόρησης}

Είναι γνωστό ότι το μεγαλύτερο μέρος του αστικού πληθυσμού είναι άτομα τα
οποία έχουν καθημερινές υποχρεώσεις και πρέπει να βρίσκονται
σε συγκεκριμένα μέρη σε συγκεκριμένες ώρες όπως φοιτητές και εργαζόμενοι.
Είναι εύκολο να δει κανείς πως τα περισσότερα από αυτά τα άτομα
έχουν κοινό προορισμό και μάλιστα μετακινούνται και σε κοινές ώρες της
ημέρας.

Ανάμεσα στον μεγάλο αριθμό ατόμων με κοινό προορισμό πολλοί επιλέγουν
την αυτοκίνηση λόγω της κακής κατάστασης των μέσων μαζικής μεταφοράς.
Χαρακτηριστικό παράδειγμα είναι τα λεωφορεία με δρομολόγια προς το
πανεπιστήμιο τα οποία είναι γεμάτα τις πρωινές ώρες.
Η αύξηση της χρήσης των αυτοκινήτων οδηγεί σε αύξηση της
κυκλοφοριακής συμφόρησης και της ρύπανσης.

\subsection{Λειτουργίες της Εφαρμογής}

Η εφαρμογή μας στοχεύει στην επίλυση αυτού του προβλήματος συνδυάζοντας την
λειτουργία του carpooling με την εύρεση καλύτερων θέσεων στάθμευσης.

\subsubsection{Carpooling και Συνδυασμός Διαδρομών}

Η εφαρμογή θα επιτρέπει στους χρήστες να δηλώνουν τις δραστηριότητες τους
και να αναζητήσουν άλλους χρήστες που πηγαίνουν στον ίδιο προορισμό.
Μέσου αυτού του μηχανισμού οι χρήστες θα μπορούν να μοιράζονται το
ίδιο όχημα και να μειώνουν το κόστος της μεταφοράς τους καθώς και το
περιβαλλοντικό τους αποτύπωμα ενώ ταυτόχρονα θα μειώνεται η κυκλοφοριακή
συμφόρηση στους δρόμους.

\subsection{Σενάρια}

Ο χρήστης κατεβάζει την εφαρμογή από το Play Store ή το App Store.
Έπειτα δημιουργεί έναν λογαριασμό μέσω email, Google ή Facebook.
Αφού συνδεθεί στην εφαρμογή μπορεί να δηλώσει τις δραστηριότητες του.

Η έννοια της δραστηριότητας είναι γενική και μπορεί να αφορά οτιδήποτε
από την εργασία του χρήστη, την σχολή του ή κάποια άλλη υποχρέωση.
Συγκεκριμένα δηλώνει μια τοποθεσία και τις ώρες άνα μέρα της εβδομάδας
για κάθε δραστηριότητα. Επιπλέον δηλώνει εάν διαθέτει όχημα ή όχι.

Έχοντας δηλώσει την δραστηριότητα του μπορεί να αναζητήσει άλλους χρήστες
οι οποίοι έχουν δηλώσει όμοιες
\footnote{
    Η έννοια της ομοιότητας αποτελεί σχεδιαστική λεπτομέρεια αλλά
    αναμένεται να βασίζεται σε γεωγραφικές αποστάσεις και χρονικές διαφορές.
}
δραστηριότητες και να κανονίσει να μοιραστεί ένα όχημα μαζί τους για
την κοινή τους διαδρομή.

Ένας εκ των χρηστών που διαθέτουν όχημα επιλέγεται
ως ο οδηγός της διαδρομής και οι υπόλοιποι ως επιβάτες. Ο οδηγός μπορεί να
επιλέξει να μοιραστεί το κόστος της διαδρομής με τους επιβάτες. Οι
επιβάτες μπορούν να αποδεχτούν ή να απορρίψουν την προσφορά του οδηγού.
Στην οποία περίπτωση η διαδικασία επαναλαμβάνεται.
Αφού συμφωνηθεί η συνεπιβίβαση επιλέγονται η ώρα και το σημείο συνάντησης
για την παραλαβή των επιβατών μέσω της εφαρμογής.

Σε κάθε παράληψη επιβάτη υπάρχει επιβεβαίωση μέσω της εφαρμογής και ο
οδηγός λαμβάνει προσωρινά πόντους για την επιβράβευση του οι οποίοι
μονιμοποιούνται όταν ολοκληρωθεί η διαδρομή.

Αφού ολοκληρωθεί η διαδρομή οι επιβάτες μπορούν να βαθμολογήσουν τον
οδηγό και να αφήσουν σχόλια για την κατάσταση του οχήματος και την
συμπεριφορά του οδηγού. Ομοίως ο οδηγός μπορεί να βαθμολογήσει τους
επιβάτες.

\newpage

\section{Domain Model}

\begin{description}
    \item[User]
        Οποιοσδήποτε χρήστης της εφαρμογής. Με αυτόν τον όρο περιγράφουμε
        οποιοδήποτε άτομο, επιχείρηση ή άλλη οντότητα που αξιοποιεί τις
        υπηρεσίες της εφαρμογής.
    \item[Driver]
        Ένας χρήστης της εφαρμογής ο οποίος οδηγεί κάποιο όχημα.
    \item[Vehicle]
        Ένα όχημα που χρησιμοποιείται για μετακίνηση από κάποιον οδηγό της
        εφαρμογής.
    \item[Rating]
        Η βαθμολογία που δίνει ένας χρήστης σε έναν άλλο χρήστη. Μπορεί να
        περιλαμβάνει σχόλια και εικόνες.
    \item[Report]
        Μια αναφορά που κάνει ένας χρήστης σε έναν άλλο χρήστη για να
        καταγγείλει παράνομη η ανεπιθύμητη δραστηριότητα. Περιλαμβάνει
        κείμενο και έναν λόγο αναφοράς.
    \item[Location]
        Μια γεωγραφική τοποθεσία, περιλαμβάνει διεύθυνση ή συντεταγμένες.
    \item[Route]
        Μια διαδρομή από ένα σημείο Α σε ένα σημείο Β. Μπορεί να περιλαμβάνει
        σημεία στάσης.
    \item[Ride]
        Αντιπροσωπεύει μια συνεπιβίβαση. Ένας οδηγός εκτελώντας μια διαδρομή
        μπορεί να μεταφέρει περισσότερους από έναν επιβάτες οι οποίοι έχουν
        κοινό προορισμό.
    \item[Pickup]
        Αντιπροσωπεύει την παραλαβή ενός επιβάτη από τον οδηγό. Περιλαμβάνει
        σημείο συνάντησης και ώρα.
    \item[Carpooler]
        Ένας χρήστης της εφαρμογής ο οποίος δεν οδηγεί κάποιο όχημα αλλά
        χρησιμοποιεί τις υπηρεσίες της εφαρμογής για να βρει οδηγούς που πάνε
        στο ίδιο μέρος με αυτόν.
    \item[Area]
        Μια περιοχή (πχ ένας δήμος).
    \item[Reward]
        Μία απολαβή είδους κουπονιού-έκπτωσης την οποία μπορεί να εξαργυρώσει
        ο χρήστης.
    \item[Activity]
        Μια δραστηριότητα είναι κάτι το οποίο θέλει να κάνει ο χρήστης σε
        συγκεκριμένο μέρος και ώρα και για το οποίο χρειάζεται μεταφορικό μέσο.
\end{description}

\begin{figure}
    \centering
    \includegraphics[width=\textwidth]{uml/domain-model}
    \caption{Domain Model}
\end{figure}

\newpage

\section{Use Cases}

\subsection{Find Ride}

Ο χρήστης επιθυμεί να βρεί οδηγό με κοινή διαδρομή με αυτόν για να συμμετέχει σε
κάποια δραστηριότητα (εργασία, μάθημα κλπ).

\subsubsection{Βασική Ροή}
\begin{enumerate}
    \item Ο χρήστης επιλέγει "Find Ride"
    \item Η εφαρμογή εμφανίζει τις δραστηριότητες του χρήστη.
    \item Ο χρήστης επιλέγει μια δραστηριότητα.
    \item To σύστημα εκτελεί αναζήτηση με βάση τα κριτήρια του χρήστη για δραστηριότητες
          άλλων χρηστών που εμφανίζουν τοπική και χρονική ομοιότητα.
    \item Το σύστημα κατατάσει τις δραστηριότητες με βάση την ομοιότητα.
    \item Το σύστημα εμφανίζει τις δραστηριότητες που βρέθηκαν.
    \item Ο χρήστης επιλέγει μια δραστηριότητα.
    \item Το σύστημα εμφανίζει τα στοιχεία του χρήστη που έχει δηλώσει την δραστηριότητα.
    \item Ο χρήστης επιλέγει "Pool".
    \item Το σύστημα ειδοποιεί τον δέκτη και ο χρήστης περιμένει την επιβεβαίωση του.
    \item O δέκτης αποδέχεται την πρόταση του χρήστη.
    \item Το σύστημα ενημερώνει τον χρήστη για τον επιτυχή προγραμματισμό.
\end{enumerate}

\subsubsection{Εναλλακτική Ροή: Ακύρωση 1}

\begin{enumerate}
    \item[3] Ο χρήστης ακυρώνει την αναζήτηση και επιστρέφει στην αρχική οθόνη.
\end{enumerate}

\subsubsection{Εναλλακτική Ροή: Ακύρωση 2}

\begin{enumerate}
    \item[7] Ο χρήστης ακυρώνει την αναζήτηση και επιστρέφει στην αρχική οθόνη.
\end{enumerate}

\subsubsection{Εναλλακτική Ροή: Ακύρωση 3}

\begin{enumerate}
    \item[9] Ο χρήστης αλλάζει γνώμη και πατάει "επιστροφή".
    \item[10] Συνέχεια από το βήμα 6 της βασικής ροής.
\end{enumerate}

\subsubsection{Εναλλακτική Ροή: Απόρριψη Πρότασης}

\begin{enumerate}
    \item[11] Ο δέκτης απορρίπτει την πρόταση του χρήστη.
    \item[12] Συνέχεια από το βήμα 6 της βασικής ροής.
\end{enumerate}

\subsubsection{Εναλλακτική Ροή: Εσωτερικό Σφάλμα}

\begin{enumerate}
    \item[5] Προκύπτει εσωτερικό σφάλμα κατά την αναζήτηση.
    \item[6] Το σύστημα ενημερώνει τον χρήστη για το σφάλμα και προτρέπει τον
        χρήστη σε αναφορά σφάλματος.
\end{enumerate}

\subsubsection{Εναλλακτική Ροή: Ο χρήστης δεν έχει δραστηριότητες}

\begin{enumerate}
    \item[2] Η εφαρμογή προτείνει την δημιουργία μιας δραστηριότητας.
    \item[3] Συνέχεια από το βήμα 1 της βασικής ροής του "Create Activity".
\end{enumerate}

\subsubsection{Εναλλακτική Ροή: Δεν βρέθηκαν όμοιες δραστηριότητες}

\begin{enumerate}
    \item[5] Το σύστημα δεν βρίσκει καμία δραστηριότητα που να ταιριάζει με τα
        κριτήρια του χρήστη.
    \item[6] Το σύστημα ενημερώνει τον χρήστη για την αποτυχία και προτείνει
        τροποποίηση της δραστηριότητας ή τη χρήση δημόσιας συγκοινωνίας.
    \item[7] Ο χρήστης επιλέγει "ΟΚ" και επιστρέφει στην αρχική οθόνη.
\end{enumerate}


\subsection{Create Activity}

Ο χρήστης επιθυμεί να δημιουργήσει μια δραστηριότητα στην εφαρμογή.

\subsubsection{Βασική Ροή}

\begin{enumerate}
    \item Ο χρήστης επιλέγει "Create Activity"
    \item Η εφαρμογή δημιουργεί μια κενή δραστηριότητα και την αποθηκεύει
          προσωρινά.
    \item Συνέχεια από το βήμα 2 της βασικής ροής του use case "Edit Activity".
\end{enumerate}


\subsubsection{Edit Activity}

\begin{enumerate}
    \item Ο χρήστης επιλέγει "Edit Activity".
    \item Η εφαρμογή εμφανίζει μενού με επιλογές για την ιδιότητα του χρήστη (πχ φοιτητής).
    \item Ο χρήστης επιλέγει την ιδιότητα του.
    \item Η εφαρμογή εμφανίζει την φόρμα αναζήτησης και έναν χάρτη της περιοχής του χρήστη.
    \item Ο χρήστης δηλώνει την περιοχή στην οποία επιθυμεί να μετακινηθεί.
    \item Η εφαρμογή εμφανίζει μενού με επιλογές για τις μέρες και τις ώρες έναρξης και λήξης
          της δραστηριότητας.
    \item Ο χρήστης εισάγει τα κατάλληλα στοιχεία.
    \item H εφαρμογή εμφανίζει μενού με επιλόγες για το μέσο μεταφοράς του χρήστη
    \item Ο χρήστης επιλέγει το μέσο μεταφοράς του.
    \item Το σύστημα εκτελεί προεπεξεργασία στα δεδομένα.
    \item Το σύστημα εισάγει την δραστηριότητα στο κατάλογο δραστηριοτήτων του χρήστη.
    \item Η εφαρμογή εμφανίζει μήνυμα επιτυχίας.
\end{enumerate}

\subsubsection{Εναλλακτική Ροή: Ακύρωση 1}

\begin{enumerate}
    \item[3] Ο χρήστης ακυρώνει την διαδικασία και επιστρέφει στην αρχική οθόνη.
\end{enumerate}

\subsubsection{Εναλλακτική Ροή: Ακύρωση 2}

\begin{enumerate}
    \item[5] Ο χρήστης ακυρώνει την διαδικασία και επιστρέφει στην αρχική οθόνη.
\end{enumerate}

\subsubsection{Εναλλακτική Ροή: Ακύρωση 3}

\begin{enumerate}
    \item[7] Ο χρήστης ακυρώνει την διαδικασία και επιστρέφει στην αρχική οθόνη.
\end{enumerate}

\subsubsection{Εναλλακτική Ροή: Ακύρωση 4}

\begin{enumerate}
    \item[9] Ο χρήστης ακυρώνει την διαδικασία και επιστρέφει στην αρχική οθόνη.
\end{enumerate}

\subsubsection{Εναλλακτική Ροή: Μη έγκυρα στοιχεία}

\begin{enumerate}
    \item[7] Ο χρήστης εισάγει μη έγκυρα στοιχεία.
    \item[8] Το σύστημα εμφανίζει μήνυμα σφάλματος και ζητά διόρθωση.
    \item[9] Συνέχεια από το βήμα 6 της βασικής ροής.
\end{enumerate}

\subsection{Manage Account}
\label{uc:manage-account}

Ο χρήστης επιθυμεί να ελέγξει ή να ενημερώσει τα στοιχεία του λογαριασμού του.

\subsubsection{Βασική Ροή}

\begin{enumerate}
    \item[1] Ο χρήστης μπαίνει στην ενότητα "Account"
    \item[2] O χρήστης επιλέγει "Profile".
    \item[3] Το σύστημα εμφανίζει τα τρέχοντα προσωπικά στοιχεία.
    \item[4] Ο χρήστης τροποποιεί ένα ή περισσότερα στοιχεία.
    \item[5] Το σύστημα εκτελεί έλεγχο των στοιχείων.
    \item[6] To σύστημα ενημερώνει τα στοιχεία και αποθηκεύει τις αλλαγές.
\end{enumerate}

\subsubsection{Εναλλακτική Ροή: Λανθασμένα Στοιχεία}

\begin{enumerate}
    \item[6] Το σύστημα εμφανίζει μήνυμα σφάλματος και ζητά διόρθωση των στοιχείων.
    \item[7] Ο χρήστης διορθώνει τα στοιχεία.
    \item[8] Συνέχεια από το βήμα 5 της βασικής ροής.
\end{enumerate}

\subsubsection{Εναλλακτική Ροή: Έξοδος}

\begin{enumerate}
    \item[2] Ο χρήστης επιλέγει το πλήκτρο επιστροφής.
    \item[3] Η εφαρμογή επιστρέφει στην κεντρική οθόνη.
\end{enumerate}

\subsubsection{Εναλλακτική Ροή: Προβολή Ιστορικού}

\begin{enumerate}
    \item[2] Ο χρήστης επιλέγει "History".
    \item[3] Το σύστημα εμφανίζει μια λίστα των συμβάντων \textit{Parking} και
        \textit{Carpooling} στα οποία έχει συμμετάσχει ο χρήστης στο παρελθόν.
    \item[4] Ο χρήστης επιλέγει να δει λεπτομέρειες για κάποιο συμβάν.
    \item[5] Το σύστημα εμφανίζει τις λεπτομέρειες του συμβάντος.
    \item[6] Ο χρήστης επιλέγει το πλήκτρο επιστροφής.
    \item[7] Καλείται η περίπτωση χρήσης \nameref{uc:manage-account}.
\end{enumerate}

\subsubsection{Εναλλακτική Ροή: Διαγραφή Συμβάντος από το Ιστορικό}

\begin{enumerate}
    \item[2] Ο χρήστης επιλέγει "History".
    \item[3] Το σύστημα εμφανίζει μια λίστα των συμβάντων \textit{Parking} και
        \textit{Carpooling} στα οποία έχει συμμετάσχει ο χρήστης στο παρελθόν.
    \item[4] Ο χρήστης επιλέγει να διαγράψει το συμβάν από το ιστορικό.
    \item[5] Το σύστημα ζητά επιβεβαίωση.
    \item[6] Ο χρήστης επιβεβαιώνει τη διαγραφή.
    \item[7] Το σύστημα διαγράφει το συμβάν από το ιστορικό.
    \item[8] Το σύστημα εμφανίζει μήνυμα επιτυχίας.
    \item[9] Καλείται η περίπτωση χρήσης \nameref{uc:manage-account}.
\end{enumerate}

\subsubsection{Εναλλακτική Ροή: Προβολή \textit{Rating}}

\begin{enumerate}
    \item[2] Ο χρήστης επιλέγει "Rating".
    \item[3] Το σύστημα εμφανίζει την τρέχουσα βαθμολογία του χρήστη και τυχόν
        σχόλια από άλλους χρήστες.
    \item[4] Ο χρήστης επιλέγει το πλήκτρο επιστροφής.
    \item[5] Καλείται η περίπτωση χρήσης \nameref{uc:manage-account}.
\end{enumerate}


% TODO
% \include{tex/use-cases/view-history}
% \include{tex/use-cases/view-rating}
\hypertarget{rate-user}{%
\subsection{Rate User}\label{rate-user}}

\hypertarget{ux3c0ux3b5ux3c1ux3b9ux3b3ux3c1ux3b1ux3c6ux3ae}{%
\subsubsection{Περιγραφή}\label{ux3c0ux3b5ux3c1ux3b9ux3b3ux3c1ux3b1ux3c6ux3ae}}

Ο χρήστης επιθυμεί να βαθμολογήσει έναν άλλο χρήστη.

\hypertarget{ux3b2ux3b1ux3c3ux3b9ux3baux3ae-ux3c1ux3bfux3ae}{%
\paragraph{Βασική
Ροή}\label{ux3b2ux3b1ux3c3ux3b9ux3baux3ae-ux3c1ux3bfux3ae}}

\begin{enumerate}
\def\labelenumi{\arabic{enumi}.}
\tightlist
\item
  Ο χρήστης επιλέγει ``Rate'' στην οθόνη User Details.
\item
  Το σύστημα ελέγχει αν υπάρχει ήδη βαθμολογία στο προφίλ του
  βαθμολογούμενου.
\item
  Η εφαρμογή εμφανίζει την φόρμα βαθμολόγησης στην οθόνη Rate User.
\item
  Ο χρήστης επιλέγει την βαθμολογία που θεωρεί και προσθέτει σχόλια στην
  οθόνη Rate User.
\item
  Το σύστημα εμφανίζει τον διάλογο επιβεβαίωσης βαθμολογίας.
\item
  Ο χρήστης επιλέγει ``Confirm'' στον διάλογο επιβεβαίωσης.
\item
  Το σύστημα καταχωρεί την βαθμολογία στο προφίλ του βαθμολογούμενου.
\item
  Το σύστημα υπολογίζει τη νέα συνολική βαθμολογία του βαθμολογούμενου.
\item
  Το σύστημα εμφανίζει μήνυμα επιτυχίας στην οθόνη User Details.
\end{enumerate}

\hypertarget{ux3b5ux3bdux3b1ux3bbux3bbux3b1ux3baux3c4ux3b9ux3baux3ae-ux3c1ux3bfux3ae-ux3bf-ux3c7ux3c1ux3aeux3c3ux3c4ux3b7ux3c2-ux3adux3c7ux3b5ux3b9-ux3b4ux3ceux3c3ux3b5ux3b9-ux3b2ux3b1ux3b8ux3bcux3bfux3bbux3bfux3b3ux3afux3b1}{%
\paragraph{Εναλλακτική Ροή: Ο χρήστης έχει δώσει
βαθμολογία}\label{ux3b5ux3bdux3b1ux3bbux3bbux3b1ux3baux3c4ux3b9ux3baux3ae-ux3c1ux3bfux3ae-ux3bf-ux3c7ux3c1ux3aeux3c3ux3c4ux3b7ux3c2-ux3adux3c7ux3b5ux3b9-ux3b4ux3ceux3c3ux3b5ux3b9-ux3b2ux3b1ux3b8ux3bcux3bfux3bbux3bfux3b3ux3afux3b1}}

\begin{enumerate}
\def\labelenumi{\arabic{enumi}.}
\setcounter{enumi}{2}
\tightlist
\item
  H εφαρμογή εμφανίζει μήνυμα ήδη υπάρχουσας βαθμολογίας στην οθόνη User
  Details.
\end{enumerate}

\hypertarget{ux3b5ux3bdux3b1ux3bbux3bbux3b1ux3baux3c4ux3b9ux3baux3ae-ux3c1ux3bfux3ae-ux3b1ux3baux3cdux3c1ux3c9ux3c3ux3b7}{%
\paragraph{Εναλλακτική Ροή:
Ακύρωση}\label{ux3b5ux3bdux3b1ux3bbux3bbux3b1ux3baux3c4ux3b9ux3baux3ae-ux3c1ux3bfux3ae-ux3b1ux3baux3cdux3c1ux3c9ux3c3ux3b7}}

\begin{enumerate}
\def\labelenumi{\arabic{enumi}.}
\setcounter{enumi}{5}
\tightlist
\item
  Ο χρήστης επιλέγει ``Cancel'' στον διάλογο επιβεβαίωσης.
\item
  Το σύστημα επιστρέφει στην οθόνη User Details.
\end{enumerate}

\hypertarget{ux3b1ux3bdux3acux3bbux3c5ux3c3ux3b7-ux3b5ux3c5ux3c1ux3c9ux3c3ux3c4ux3afux3b1ux3c2}{%
\subsubsection{Ανάλυση
Ευρωστίας}\label{ux3b1ux3bdux3acux3bbux3c5ux3c3ux3b7-ux3b5ux3c5ux3c1ux3c9ux3c3ux3c4ux3afux3b1ux3c2}}

\begin{figure}
\centering
\includegraphics{./rate-user-robustness.drawio.png}
\caption{image}
\end{figure}

\subsection{Report User}

Ο χρήστης επιθυμεί να αναφέρει έναν άλλο χρήστη της εφαρμογής για
παράνομη ή ανεπιθύμητη δραστηριότητα.

\subsubsection{Βασική Ροή}

\begin{enumerate}
    \item Ο χρήστης επιλέγει τον χρήστη που θέλει να αναφέρει και πατάει "Report".
    \item H εφαρμογή εμφανίζει την φόρμα αναφοράς.
    \item Ο χρήστης επιλέγει τον λόγο αναφοράς, προσθέτει σχόλια και πατάει υποβολή.
    \item Η εφαρμογή ενημερώνει τον χρήστη για την επιτυχή υποβολή.
    \item Το σύστημα επεξεργάζεται την αναφορά και την τοποθετεί σε λίστα αναμονής.
\end{enumerate}

\subsubsection{Εναλλακτική Ροή: Ακύρωση}

\begin{enumerate}
    \item[3] Ο χρήστης αποφασίζει να μην υποβάλει την αναφορά και πατάει "Cancel".
\end{enumerate}

\subsection{Redeem Reward}

Ο χρήστης επιθυμεί να λάβει μια ανταμοιβή που έχει κερδίσει μέσω
της εφαρμογής.

\subsubsection{Βασική Ροή}

\begin{enumerate}
    \item Ο χρήστης επιλέγει "Redeem Reward"
    \item Η εφαρμογή υπολογίζει τους πόντους του χρήστη.
    \item Η εφαρμογή εμφανίζει τις διαθέσιμες ανταμοιβές και τους πόντους.
    \item Ο χρήστης επιλέγει την ανταμοιβή που επιθυμεί.
    \item Η εφαρμογή ενημερώνει τους πόντους του χρήστη και αφαιρεί την ανταμοιβή
          από τη λίστα των διαθέσιμων ανταμοιβών.
    \item Η εφαρμογή εμφανίζει τον κωδικό εξαργύρωσης.
\end{enumerate}

\subsubsection{Εναλλακτική Ροή: Ακύρωση}

\begin{enumerate}
    \item[4] Ο χρήστης εγκαταλείπει την διαδικασία.
\end{enumerate}

\subsubsection{Εναλλακτική Ροή: Δεν υπάρχουν διαθέσιμες ανταμοιβές}

\begin{enumerate}
    \item[3] Η εφαρμογή εμφανίζει μήνυμα ότι δεν υπάρχουν διαθέσιμες ανταμοιβές.
\end{enumerate}

\subsubsection{Εναλλακτική Ροή: Ανεπαρκής αριθμός πόντων}

\begin{enumerate}
    \item[5] Η εφαρμογή εμφανίζει μήνυμα ότι ο χρήστης δεν έχει αρκετούς πόντους.
    \item[6] Συνέχεια από το βήμα 3 της βασικής ροής.
\end{enumerate}


\begin{figure}
    \centering
    \includegraphics[width=0.8\textwidth]{uml/use-cases}
    \caption{Use Case Diagram}
\end{figure}

\section{Απαιτήσεις και Προδιαγραφές}

\section{Χρήση Τεχνολογιών και Εργαλείων}

\end{document}
